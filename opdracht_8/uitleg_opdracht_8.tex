\documentclass{article}
\usepackage{enumitem}
\usepackage{Sweave}
\begin{document}
\Sconcordance{concordance:uitleg_opdracht_8.tex:uitleg_opdracht_8.Rnw:%
1 2 1 1 0 42 1}



\section{Simulties opdracht 7}
\label{sec:sample_main}

\subsection{Opdrachtomschrijving}
\label{subsec:sample_opdrom}

Met simulaties kun je ook oppervlaktes en integralen bepalen. Hieronder zie je bijvoorbeeld de grafiek van op het interval .

De oppervlakte onder die grafiek is niet exact te bepalen door primitiveren en dus moet de oppervlakte benaderd worden. Dat kan als volgt met een simulatie:

\begin{enumerate}[label=(\Alph*)]
\item Neem een rechthoek om de te bepalen oppervlakte

\item Genereer n willekeurige (x,y) in dit oppervlakte

\item Tel het aantal punten m dat in de gevraagde oppervlakte ligt;

\item Bereken de verhouding , dit getal is een benadering van de oppervlakteverhouding. Als n naar oneindig gaat dan convergeert m/n*opp(vierkant) naar de oppervlakte.

\item Voer deze simulatie uit. Vergelijk de uitkomst van de simulatie ook met een numerieke berekening van de waarde van de integraal.
\end{enumerate}

\newpage
\subsection{Uitleg}
\label{subsec:sample_uitv}

In het bijbehorende R file word je aan de hand meegenomen naar een oplossing voor dit probleem.

Ook werd er gevraagd een numerieke berekening van de waarde van de integraal te bepalen. Dit heb ik met deze functie gedaan
\begin{verbatim}
integrate(sin(x^2), {x,0,3/2})
\end{verbatim}

De waarde uit deze berekening is 0.778
\section{Conclusie}
\label{subsec:sample_con}
De oppervlakte onder de grafiek berekend door mijn simulatie was 0.76. Het antwoord vanuit mathematica is 0.78. De simulatie is een goede benadering voor het probleem.

\end{document}
