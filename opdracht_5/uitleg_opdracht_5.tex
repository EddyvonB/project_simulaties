\documentclass{article}
\usepackage{enumitem}
\usepackage{Sweave}
\begin{document}
\Sconcordance{concordance:uitleg_opdracht_5.tex:uitleg_opdracht_5.Rnw:%
1 2 1 1 0 45 1}



\section{Simulties opdracht 4}
\label{sec:sample_main}

\subsection{Opdrachtomschrijving}
\label{subsec:sample_opdrom}

Probleem:

Hoe groot is de kans dat de kwadratische vergelijking geen nulpunten heeft als de coëfficiënten a, b en c willekeurig uit het continue interval [-20,20] gekozen worden?

Ook dit probleem is niet eenvoudig analytisch op te lossen. Met een simulatie kun je deze kans schatten.


\begin{enumerate}[label=(\Alph*)]

\item Maak een werkblad in Excel waarmee je dit probleem simuleert. Laat je inspireren door het bestand stok.xls. Schat de kans op 2 decimalen nauwkeurig.

\item Onderzoek ook hoe groot de kans is als de coefficienten [-30,30] in liggen.

Waarschijnlijk heb je in onderdeel b) de waarde in een bepaalde cel veranderd. Als je dat vele malen wilt herhalen dan kan dat het beste met een schuifbalk in Excel. Een gebruiker kan dan een waarde continue veranderend instellen. Zoek eens uit hoe je in Excel een schuifbalk maakt (tabblad ontwikkelaars toevoegen, kies een formulierbesturingselement).

\item Maak nu een schuifbalk waarmee je de maximale waarde voor a, b, c kunt instellen van 20 (waarde onderdeel a) tot 100.

Gebruik de schuifbalk voor de volgende vragen:

\item Hoe groot is de kans op geen oplossingen als de coëfficiënten [-60,60] in liggen?

\item Hoe groot wordt die kans als de maximumwaarde steeds groter wordt. Convergeert de kans naar een vaste waarde?
\end{enumerate}


\newpage
\subsection{Uitleg}
\label{subsec:sample_uitv}

\section{Conclusie}
\label{subsec:sample_con}




\end{document}
