\documentclass{article}
\usepackage{enumitem}
\usepackage{Sweave}
\begin{document}
\Sconcordance{concordance:uitleg_opdracht_2.tex:uitleg_opdracht_2.Rnw:%
1 2 1 1 0 60 1}



\section{Simulties opdracht 2}
\label{sec:sample_main}

\subsection{Opdrachtomschrijving}
\label{subsec:sample_opdrom}

Met behulp van ASELECT() kun je randomgetallen uit kansverdelingen genereren. In de cursus spreadsheets Excel 1 heb je bv. geleerd dat AFRONDEN.BOVEN(6*ASELECT();1) de gehele getallen 1 t/m 6 met gelijke kans genereert. Je "trekt" dan een randomgetal uit de discrete uniforme verdeling met 6 elementen.

Maak in Excel functies die randomgetallen voortbrengen uit:
\begin{enumerate}[label=(\Alph*)]
\item de discrete uniforme verdeling met waarden 0 t/m 13  
\item de uniforme continue verdeling met grenzen 0 en 10
\item de uniforme continue verdeling met grenzen -2 en 3
\item de discrete kansverdeling met verdeling:
\end{enumerate}

\begin{table}[h]
  \begin{tabular}{|l|r|r|}
      \hline
      k & 0 & 1 \\ \hline
      kans & 2/3 & 1/3 \\ 
      \hline
  \end{tabular}
\end{table}

\newpage
\subsection{Uitvoering}
\label{subsec:sample_uitv}

In de bijgevoegde excel is opdracht 2 uitgewerkt. Deze sectie geeft toelichting waar nodig en de stappen die doorlopen zijn.

\begin{enumerate}[label=(\Alph*)]
\item in kolom b is de volgende functie gebruikt: \begin{equation} ROUNDDOWN(14*RAND();0) \end{equation}  Er is gekozen voor rounddown omdat de functie rand 1 niet meeneemt en 13 moet er wel bij zitten. Omdat de functie rand uniform verdeeld is en de waardes discreet zijn, krijg je hier een discrete uniforme verdeling.
\item In kolom e is de functie:\begin{equation}  \frac{RANDBETWEEN(0*1000000;10*1000000)}{1000000} \end{equation} gebruikt. Hiermee krijg je uniform continue verdeling. Elke waarde tussen 0 en 10 kan met deze functie gegeneerd worden, daardoor is deze genereerd deze functie continue variablen. Omdat op elke variable evenveel kans is, is deze ook uniform verdeelt.
\item In kolom h is de functie: \begin{equation} \frac{RANDBETWEEN(-2*1000000;3*1000000)}{1000000} \end{equation} gebruikt. Hiermee krijg je uniform continue verdeling. Elke waarde tussen -2 en 3 kan met deze functie gegeneerd worden, daardoor is deze genereerd deze functie continue variablen. Omdat op elke variable evenveel kans is, is deze ook uniform verdeelt.
\item Voor opdracht D) heb is er een tabel gegeneerd die als volgt is:

\begin{table}[h]
  \begin{tabular}{|l|r|r|}
      \hline
      k & 0 & 1 \\ \hline
      kans & 2/3 & 1/3 \\ 
      \% & 0\% & 67\% \\ 
      \hline
  \end{tabular}
\end{table}

Met de derde rij van deze tabel kunnen we de volgende functie gebruiken \begin{equation} LOOKUP(RAND();0\%:67\%;0:1) \end{equation} Deze functie geeft met kans 67\% 0 terug en met 33\% kans 1 terug. Dit is conform de wens bij D) in kolom k van het bijgevoegde excel bestand zie je de resultaten.
\end{enumerate}

\section{Conclusie}
\label{subsec:sample_con}

Aan opdracht 2 is aan alle eisen voldaan. Echter opdracht B) en C) zouden nog uitgebreider kunnen als je het getal wat nu 1.000.000 is, groter maakt.


\end{document}
